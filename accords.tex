\documentclass[11pt]{article}

\usepackage[T1]{fontenc}
\usepackage[utf8]{inputenc}
\usepackage[francais]{babel}
\usepackage{fancyhdr}
\usepackage{datetime} % access to \currenttime
\usepackage{hyperref}
\usepackage[chorded]{songs}

\setlength{\parskip}{2ex}

\title{\vspace{-5em}Les accords de la guitare}
\author{(accordage standard)}
\date{} 

\pagestyle{fancy}
\fancyhead[R]{\today~-~\currenttime}
\fancyfoot[C]{\thepage}

\begin{document}

\maketitle

\setcounter{tocdepth}{2}
\tableofcontents

\section{À propos de ce document}

Ce document est une présentation progressive des accords que l’on peut jouer
sur une guitare.

Attention, ce n’est ni un cours de guitare, ni un cours d’harmonie.
Ça serait plutôt un support pour un cours de guitare. On n’y explique pas en
détail la position des notes sur le manche, les doigtés, le rôle des notes dans
l’accord, les cadences… Pour cela, voir la section \ref{approf}.

Les accords sont présentés dans une progression à la fois technique (les plus
faciles à jouer au début) et théorique (les plus simples au début), avec
quelques exceptions arbitraires, si elles ne sont pas pédagogiques.

On se restreint à l’accordage standard~: Mi grave, La, Ré, Sol, Si et Mi aigu
(de la sixième corde à la première corde).

Voir la section \ref{contact} à la fin de ce document pour plus de détails ou
partager vos idées.

\section{Premiers accords sans basse}

\gtab{Sol7}{XXX001} est parfois le premier présenté.
\hspace{2em}
\gtab{Do}{XXX010} est sa réponse.

Ça n’a l’air de rien, mais avec ces deux accords et un capodastre, on peut
jouer pas mal de chansons.

Les diagrammes seront tous dessinés pour les droitiers~:

\begin{itemize}
\item le trait vertical sur la gauche est la corde de Mi grave~;
\item le trait vertical sur la droite est la corde de Mi aigu~;
\item le trait horizontal tout en haut est le sillet~;
\item les autres traits horizontaux sont les frettes.
\end{itemize}

Attention, comme dit dans l’introduction, la corde désignée comme étant la
«~première corde~» est la corde la plus aigue.

La première case, située entre le sillet et la première frette, est
dessinée entre les deux premiers traits horizontaux~:

\gtab{cordes}{000000:654321}
\gtab{1ère case}{111111}

Nous reviendrons aux accords sans basse plus loin.

\section{Accords de puissance}

En anglais~: «~power chords~».

Ils sont notés avec un 5 car en plus de la note qui donne son nom à l’accord,
on ajoute seulement la quinte. Du coup, ils conviennent à la fois pour les
accords majeurs et les accords mineurs.

\subsection{Premier accord de puissance}

\gtab{Do5}{X35XXX:013000}

Utiliser le petit doigt à la place de l’annulaire si c’est trop difficile avec
l’annulaire. Mais si vous arrivez à le jouer avec l’annulaire, vous pourrez
ajouter une note~:

\gtab{Do5}{X355XX:013400}

\subsection{Les sept principaux}

\gtab{Do5}{X355XX}
\gtab{Ré5}{\lower -3pt\hbox{5}5:X133XX:000000}
\gtab{Mi5}{022XXX:023000}
\gtab{Fa5}{133XXX}
\gtab{Sol5}{355XXX}
\gtab{La5}{X022XX:002300}
\gtab{La5}{5:133XXX}
\gtab{Si5}{2:X133XX}

On comprendra plus tard pouquoi Mi5 et La5 sont joués avec ces doigtés.

\subsection{Premiers barrés}

Pour Mi5 et La5, c’est aussi le moment d’essayer vos premiers «~barrés~»~!

\gtab{Mi5}{0(22)XXX}
\gtab{La5}{X0(22)XX}

\section{Notations}

\subsection{Nom anglo-saxon des notes}

Voici le tableau de correspondance~:

\begin{tabular}{ | c | c | c | c | c | c | c | }
    \hline
    La & Si & Do & Ré & Mi & Fa & Sol \\
    \hline
    A & B & C & D & E & F & G \\
    \hline
\end{tabular}

\subsection{Type des premiers accords}

Encore une fois, ce document n’est pas un cours et n’explique pas en détail
la construction et le rôle des accords.

S’il n’y a rien de plus que le nom de la note, l’accord est dit «~majeur~».
«~Do~» ou «~C~» désigne un accord de «~Do majeur~» ou tout simplement «~Do~».

Un «~m~» signifie «~mineur~». «~Dm~» signifiera donc «~Ré mineur~».

Le chiffre «~7~» se prononce simplement «~sept~», ou parfois «~septième de
dominante~». «~G7~» désigne donc l’accord «~Sol 7~».
C’est un accord à quatre sons. Il est assez dissonant et
attend souvent une résolution. C’est le seul accord à quatre sons de cette
partie, d’autres seront vus plus loin.

\section{Premiers accords avec basse}

Dans cette partie, nous verrons des accords qui se jouent «~au sillet~».
La basse de l’accord sera toujours la note la plus grave dans celui-ci.

\subsection{Avec indication des doigtés}

Dans la plupart des cas, les doigts se posent «~dans l’ordre de la lecture~»
sur les diagrammes tels qu’ils sont affichés ici, c’est-à-dire de haut en bas
puis de gauche à droite. 

\gtab{Do}{X32010:032010}
\gtab{Rém}{XX0231:000231}
\gtab{Ré}{XX0232:000132}

Plus précisément, je regarde la première case du manche, entre le sillet
et la première frette (la première ligne sur un
diagramme) et je la parcours de gauche à droite. Si je vois un point,
j’y pose mon index. Le point suivant sera joué avec le majeur, soit
sur une corde plus aigue de la première case, soit sur la deuxième case,
toujours du grave vers l’aigu. Idem pour les autres doigts.

Première exception courante~: si un doigt atteint plus facilement une case.
On peut par exemple poser ce Sol(v1) au lieu de Sol(v2)~:

\gtab{Sol(v1)}{320003:210004}
\gtab{Sol(v2)}{320003:210003}

Une autre exception courante~: quand la même forme globale se retrouve dans un
autre accord et que l’on veut utiliser ces accords l’un après l’autre sans
bouger trop de doigts. On peut poser Sol(v2) mais difficile après de passer à
Sol7. On pourra donc poser Sol(v3).

\gtab{Sol(v2)}{320003:210003}
\gtab{Sol7}{320001:320001}
\gtab{Sol(v3)}{320003:320004}

Dernière exception courante~: pour consolider la mémorisation par le corps de
certaines positions, nous jouerons le Mim(v1) plutôt que le Mim(v2).

\gtab{Mi}{022100:023100}
\gtab{Mim(v1)}{022000:023000}
\gtab{Mim(v2)}{022000:012000}

Cela permet aussi quelques effets avec l’index, et revient donc à l’exception
précédente.

\subsection{Sans indication des doigtés}

À vous d’essayer toutes les façons possibles de réaliser les accords
suivants~! Ils ici sont regroupés dans ce qu’on appelle des «~tonalités~».

Disons que les accords d’une même tonalité vont bien ensemble, donc
c’est plus agréable de les travailler les uns à la suite des autres, dans
l’ordre proposé où dans un autre ordre. Parfois vous reconnaîtrez une chanson.

Il n’y en a pas autant dans chaque tonalité car on se limite pour l’instant aux
accords les plus faciles à jouer, au sillet. Avec les barrés, nous pourrons
jouer tous les accords de toutes les tonalités.

\subsubsection{Do majeur ou La mineur}

\gtab{Do}{X32010}
\gtab{Rém}{XX0231}
\gtab{Mim}{022000}
\gtab{Mim}{XX2000}
\gtab{Mi}{022100}
\gtab{Mi}{XX2100}
\gtab{Mi7}{020100}
\gtab{Fa}{XX321X}
\gtab{Fa}{1XX21X}
\gtab{Sol}{320003}
\gtab{Sol7}{320001}
\gtab{Sol5}{3X0033}
\gtab{Lam}{X02210}

\subsubsection{Ré majeur ou Si mineur}

\gtab{Ré}{XX0232}
\gtab{Mim}{022000}
\gtab{Mim}{XX2000}
\gtab{Sol}{320003}
\gtab{Sol5}{3X0033}
\gtab{La}{X02220}
\gtab{La7}{X02020}

\subsubsection{Mi majeur ou Do\shrp mineur}

\gtab{Mi}{022100}
\gtab{Mi}{XX2100}
\gtab{La}{X02220}
\gtab{Si7}{X21202}

\subsubsection{Fa majeur ou Ré mineur}

\gtab{Fa}{XX321X}
\gtab{Fa}{1XX21X}
\gtab{Sol5}{3X0033}
\gtab{Lam}{X02210}
\gtab{La}{X02220}
\gtab{La7}{X02020}
\gtab{La7}{X02223}
\gtab{La7}{X02023}
\gtab{Do}{X32010}
\gtab{Do7}{X32310}
\gtab{Rém}{XX0231}

\subsubsection{Sol majeur ou Mi mineur}

\gtab{Sol}{320003}
\gtab{Sol5}{3X0033}
\gtab{Lam}{X02210}
\gtab{Si7}{X21202}
\gtab{Do}{X32010}
\gtab{Ré}{XX0232}
\gtab{Ré7}{XX0212}
\gtab{Mim}{022000}
\gtab{Mim}{XX2000}

\subsubsection{La majeur ou Fa\shrp mineur}

\gtab{La}{X02220}
\gtab{Ré}{XX0232}
\gtab{Mi}{022100}
\gtab{Mi}{XX2100}
\gtab{Mi7}{020100}
\gtab{Mi7}{020130}

\subsubsection{Si$\flat$ majeur ou Sol mineur}

\gtab{Si&}{X13X3X}
\gtab{Dom}{X3X04X}
\gtab{Rém}{XX0231}
\gtab{Ré}{XX0232}
\gtab{Ré7}{XX0212}
\gtab{Mi&}{XX13X3}
\gtab{Fa}{XX321X}
\gtab{Fa}{1XX21X}
\gtab{Sol5}{3X0033}

\section{Accords avec barrés}

Sauf mention contraire, le barré se fait avec l’index de la main gauche. Il
joue le rôle d’un sillet mobile et permet d’atteindre des tonalités jusqu’alors
inaccessibles.

À vous de trouver tous les accords possibles en déplaçant les accords suivants
sur le manche.

\subsection{Petits barrés}

\gtab{C}{8:XX32(11)}
\gtab{Am}{5:XX3(111)}
\gtab{La7}{X0(222)3}

\subsection{Grands barrés}

\gtab{C}{8:(133211)}
\gtab{Am}{5:(133111)}
\gtab{G7}{3:(131211)}

\gtab{F}{8:X(13331)}
\gtab{Dm}{5:X(13321)}
\gtab{C7}{3:X(13131)}

Pour F et Dm, nul besoin de mettre l’index bien à plat, il suffit qu’il
frette les notes sur la première et la cinquième corde.

Si votre annulaire le veut bien, vous pouvez jouer le Fa majeur ainsi~:

\gtab{F}{8:X(1(333)1)}

\subsection{Utilisation du pouce de la main gauche}

Ces accords vus précédemment~:

\gtab{F}{(133211)}
\gtab{G}{3:(133211)}

Peuvent aussi être joués ainsi~:

\gtab{F}{1332(11):P3421}
\gtab{G}{3:1332(11):P3421}

\section{Pour bien mémoriser ces accords}

\subsection{Connaître des chansons}

C’est surtout qu’on apprend des accords pour ça.

Apprendre des accords devient bien moins pénible, voire agréable si c’est
par et pour une chanson.

\subsection{Connaître les notes qui composent un accord}

Tout cela est lié à la position des touches noires sur un clavier, par exemple
de piano. Voir un cours d’harmonie pour plus de précisions.

\subsubsection{Accords majeur}

\begin{tabular}{ | c | c | c | c | c | c | c | c | }
    \hline
    Fond. & Do & Ré      & Mi      & Fa & Sol & La      & Si \\
    \hline
    3ce   & Mi & Fa\shrp & Sol\shrp & La & Si & Do\shrp & Ré\shrp \\
    \hline
    5te   & Sol & La     & Si      & Do & Ré & Mi      & Fa\shrp \\
    \hline
\end{tabular}

Ou avec les noms anglo-saxons~:

\begin{tabular}{ | c | c | c | c | c | c | c | c | }
    \hline
    Fond. & C & D      & E      & F & G & A      & B \\
    \hline
    3ce   & E & F\shrp & G\shrp & A & B & C\shrp & D\shrp \\
    \hline
    5te   & G & A      & B      & C & D & E      & F\shrp \\
    \hline
\end{tabular}

\subsubsection{Accords mineur}

\begin{tabular}{ | c | c | c | c | c | c | c | c | }
    \hline
    Fond. & Do        & Ré & Mi & Fa        & Sol        & La & Si \\
    \hline
    3ce   & Mi$\flat$ & Fa & Sol & La$\flat$ & Si$\flat$ & Do & Ré \\
    \hline
    5te   & Sol       & La & Si & Do        & Ré        & Mi & Fa\shrp \\
    \hline
\end{tabular}

Ou avec les noms anglo-saxons~:

\begin{tabular}{ | c | c | c | c | c | c | c | c | }
    \hline
    Fond. & C        & D & E & F        & G        & A & B \\
    \hline
    3ce   & E$\flat$ & F & G & A$\flat$ & B$\flat$ & C & D \\
    \hline
    5te   & G        & A & B & C        & D        & E & F\shrp \\
    \hline
\end{tabular}

\section{Accords à quatre sons}

\subsection{Le principe}

Les notes qui composent ces accords se retrouvent de la même façon que pour
trois sons. Il y a les accords «~M7~» (parfois notés $\Delta$), «~m7~», «~7~»
et «~m7b5~» (parfois notés $\emptyset$).

\subsection{Accords au sillet}

\ldots

\subsection{Accords avec barrés}

\ldots

\subsection{Drop 2}

Ces accords sont «~sans basse~». Elle peut être assurée par une contrebasse,
la main gauche d’un claviériste…

Attention~: les chiffres sous les diagrammes indiquent les degrés des notes et
non les doigtés.

\subsubsection{Sur les quatre premières cordes}

Accords de type «~majeur 7~»

\gtab{7351}{XX33(22):007351}
\gtab{1573}{XX1(333):001573}
\gtab{3715}{XX2413:003715}
\gtab{5137}{XX(222)4:005137}

Accords de type «~7~»

\gtab{7351}{XX(2322):007351}
\gtab{1573}{XX1323:001573}
\gtab{3715}{XX2313:003715}
\gtab{5137}{XX(222)3:005137}

Accords de type «~mineur 7~»

\gtab{7351}{XX(2222):007351}
\gtab{1573}{XX1322:001573}
\gtab{3715}{XX1313:003715}
\gtab{5137}{XX2213:005137}

Accords de type «~demi-diminué~»

\gtab{7351}{XX2212:007351}
\gtab{1573}{XX1(222):001573}
\gtab{3715}{XX1312:003715}
\gtab{5137}{XX1213:005137}

\subsubsection{Sur les quatre cordes du milieu}

C’est moins pratique sur ces cordes, mais instructif quand même.

\ldots

\section{Autres accords}

\subsection{Avec des cordes à vide}

Les accords sont regroupés par tonalité, quitte à ce qu’il y ait des doublons.

\subsubsection{Do majeur ou La mineur}


\subsubsection{Ré majeur ou Si mineur}


\subsubsection{Mi majeur ou Do\shrp mineur}

\gtab{Mi5}{7:013200}
\gtab{Mi $\Delta$}{7:013200}
\gtab{La sus2}{7:X01300}
\gtab{La sus2}{7:X03300}
\gtab{La add2}{6:X02100}
\gtab{La M9}{6:X01100}
\gtab{La 69}{4:X01300}
\gtab{La sus2 \shrp 11}{7:X01200}
\gtab{La sus2 \shrp 11}{7:X03200}
\gtab{Do \shrp m}{7:2X2200}
\gtab{Do \shrp m9}{7:2X2100}

\subsubsection{Fa majeur ou Ré mineur}


\subsubsection{Sol majeur ou Mi mineur}


\subsubsection{La majeur ou Fa\shrp mineur}


\subsubsection{Si$\flat$ majeur ou Sol mineur}


\subsection{TODO}

\gtab{E&69}{5:X2(11)22:020034}

Il manque encore tout un tas de trucs~:

\begin{itemize}
\item les triades (groupées ou éloignées, renversées)
\item les renversements, les basses étrangères
\item \ldots~(j’attends vos suggestions~!)
\end{itemize}

\section{Approfondissement} \label{approf}

Je propose justement dans mes cours de comprendre les gammes, la construction
des accords, le rôle des notes, des idées d’utilisation de ces accords (disons
en gros~: la main droite et des morceaux précis).

Ne pouvant pas faire mieux faute de temps, je vous propose ici d’utiliser un
moteur de recherche pour y trouver ces informations. Par exemple~:

\begin{itemize}
\item la gamme majeure~;
\item l’harmonisation de la gamme majeure~;
\item les autres gammes~;
\item des vidéos sur les rythmiques ou battements à la guitare~;
\item ne pas hésiter à chercher «~apprendre \textit{[ici mon morceau préféré]}
       à la guitare~».
\end{itemize}

\section{Contact} \label{contact}

\setlength{\parindent}{0pt}

Site~: \url{https://grahack.github.io/accords_guitare/} \\
ou plus simplement \url{https://huit.re/acc}

Contact~: \url{profgra.org@gmail.com}

\end{document}
