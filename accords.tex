\documentclass[11pt]{article}

\usepackage[T1]{fontenc}
\usepackage[utf8]{inputenc}
\usepackage[francais]{babel}
\usepackage{hyperref}
\usepackage[chorded]{songs}

\setlength{\parskip}{2ex}

\title{\vspace{-5em}Les accords de la guitare}
\author{(accordage standard)}
\date{} 

\begin{document}

\maketitle

\section{À propos de ce document}

Ce document présente des accords que l’on peut jouer sur une guitare accordée
avec l’accordage standard~: Mi grave, La, Ré, Sol, Si et Mi aigu (de la
sixième corde à la première corde).

Attention, ce n’est ni un cours de guitare, ni un cours d’harmonie.
Ça serait plutôt un support pour un cours de guitare. On n’y explique pas en
détail la position des notes sur le manche, les doigtés, le rôle des notes dans
l’accord, les cadences…

Les accords sont présentés dans une progression à la fois technique (les plus
faciles à jouer au début) et théorique (les plus simples au début), avec
quelques exceptions arbitraires, si elles ne sont pas pédagogiques.

Voir la section \ref{contact} à la fin de ce document pour plus de détails ou
partager vos idées.

\section{Premiers accords sans basse}

\gtab{Sol7}{XXX001} est parfois le premier présenté.
\hspace{2em}
\gtab{Do}{XXX010} est sa réponse.

Nous reviendrons aux accords sans basse plus loin.

\section{Accords de puissance}

En anglais~: «~power chords~».

Ils sont notés avec un 5 car en plus de la note qui donne son nom à l’accord,
on ajoute seulement la quinte. Du coup, ils conviennent à la fois pour les
accords majeurs et les accords mineurs.

\subsection{Premier accord de puissance}

\gtab{Do5}{X35XXX:013000}

Utiliser le petit doigt à la place de l’annulaire si c’est trop difficile avec
l’annulaire. Mais si vous arrivez à le jouer avec l’annulaire, vous pourrez
ajoutez une note~:

\gtab{Do5}{X355XX:013400}

\subsection{Les sept principaux}

\gtab{Do5}{X355XX}
\gtab{Ré5}{5:X133XX}
\gtab{Mi5}{022XXX:023000}
\gtab{Fa5}{133XXX}
\gtab{Sol5}{355XXX}
\gtab{La5}{X022XX:002300}
\gtab{La5}{5:133XXX}
\gtab{Si5}{2:X133XX}

On comprendra plus tard pouquoi Mi5 et La5 sont joués avec ces doigtés.

\subsection{Premiers barrés}

Pour Mi5 et La5, c’est aussi le moment d’essayer vos premiers «~barrés~»~!

\gtab{Mi5}{0(22)XXX}
\gtab{La5}{X0(22)XX}

\section{Nom anglo-saxon des notes}

\begin{tabular}{ | c | c | c | c | c | c | c | }
    \hline
    La & Si & Do & Ré & Mi & Fa & Sol \\
    \hline
    A & B & C & D & E & F & G \\
    \hline
\end{tabular}

\section{Premiers accords avec basse}

Dans cette partie, la basse de l’accord est toujours la note la plus grave.

\subsection{Avec indication des doigtés}

Dans la plupart des cas, les doigts se posent «~dans l’ordre de la lecture~»
sur les diagrammes tels qu’ils sont affichés ici, c’est-à-dire de haut en bas
puis de gauche à droite. 

\gtab{Do}{X32010:032010}
\gtab{Rém}{XX0231:000321}
\gtab{Ré}{XX0232:000132}

Plus précisément, je regarde la première case (la première ligne sur un
diagramme) de gauche à droite. Si je vois un point, j’y pose mon index.
Le point suivant sera joué avec le majeur, soit
sur une corde plus aigue de la première case, soit sur la deuxième case,
toujours du grave vers l’aigu.

Première exception courante~: si un doigt atteint plus facilement une case.
On peut par exemple poser ce Sol(v1) au lieu de Sol(v2)~:

\gtab{Sol(v1)}{320003:210004}
\gtab{Sol(v2)}{320003:210003}

Une autre exception courante~: quand la même forme globale se retrouve dans un
autre accord et que l’on veut utiliser ces accords l’un après l’autre sans
bouger trop de doigts. On peut poser Sol(v2) mais difficile après de passer à
Sol7. On pourra donc poser Sol(v3).

\gtab{Sol(v2)}{320003:210003}
\gtab{Sol7}{320001:320001}
\gtab{Sol(v3)}{320003:320004}

\subsection{Sans indication des doigtés}

À vous d’essayer toutes les façons possibles de réaliser les accords
suivants~! Ils ici sont regroupés dans ce qu’on appelle des «~tonalités~».

Disons que les accords d’une même tonalité vont bien ensemble, donc
c’est plus agréable de les travailler les uns à la suite des autres, dans
l’ordre proposé où dans un autre ordre. Parfois vous reconnaîtrez une chanson.

Il n’y en a pas autant dans chaque tonalité car on se limite pour l’instant aux
accords les plus faciles à jouer, au sillet. Avec les barrés, nous pourrons
jouer tous les accords de toutes les tonalités.

\subsubsection{Do majeur ou La mineur}

\gtab{Do}{X32010}
\gtab{Rém}{XX0231}
\gtab{Mim}{022000}
\gtab{Mim}{XX2000}
\gtab{Mi}{022100}
\gtab{Mi}{XX2100}
\gtab{Mi7}{020100}
\gtab{Fa}{XX321X}
\gtab{Fa}{1XX21X}
\gtab{Sol}{320003}
\gtab{Sol7}{320001}
\gtab{Sol5}{3X0033}
\gtab{Lam}{X02210}

\subsubsection{Ré majeur ou Si mineur}

\gtab{Ré}{XX0232}
\gtab{Mim}{022000}
\gtab{Mim}{XX2000}
\gtab{Sol}{320003}
\gtab{Sol5}{3X0033}
\gtab{La}{X02220}
\gtab{La7}{X02020}

\subsubsection{Mi majeur ou Do\shrp mineur}

\gtab{Mi}{022100}
\gtab{Mi}{XX2100}
\gtab{La}{X02220}
\gtab{Si7}{X21202}

\subsubsection{Fa majeur ou Ré mineur}

\gtab{Fa}{XX321X}
\gtab{Fa}{1XX21X}
\gtab{Sol5}{3X0033}
\gtab{Lam}{X02210}
\gtab{La}{X02220}
\gtab{La7}{X02020}
\gtab{Do}{X32010}
\gtab{Do7}{X32310}
\gtab{Rém}{XX0231}

\subsubsection{Sol majeur ou Mi mineur}

\gtab{Sol}{320003}
\gtab{Sol5}{3X0033}
\gtab{Lam}{X02210}
\gtab{Si7}{X21202}
\gtab{Do}{X32010}
\gtab{Ré}{XX0232}
\gtab{Ré7}{XX0212}
\gtab{Mim}{022000}
\gtab{Mim}{XX2000}

\subsubsection{La majeur ou Fa\shrp mineur}

\gtab{La}{X02220}
\gtab{Ré}{XX0232}
\gtab{Mi}{022100}
\gtab{Mi}{XX2100}
\gtab{Mi7}{020100}

\subsubsection{Si$\flat$ majeur ou Sol mineur}

\gtab{Si&}{X13X3X}
\gtab{Dom}{X3X04X}
\gtab{Rém}{XX0231}
\gtab{Ré}{XX0232}
\gtab{Ré7}{XX0212}
\gtab{Mi&}{XX13X3}
\gtab{Fa}{XX321X}
\gtab{Fa}{1XX21X}
\gtab{Sol5}{3X0033}

\section{Accords avec barrés}

\subsection{Petits barrés}

\gtab{G}{3:XX32(11)}
\gtab{F}{XX32(11)}

\subsection{Grands barrés}

\gtab{G}{3:(133211)}
\gtab{F}{(133211)}
\hspace{3em}
\gtab{C}{3:X(1(333)1)}
\gtab{B&}{X(1(333)1)}

\section{Pour bien mémoriser ces accords}

\subsection{Connaître des chansons}

C’est surtout qu’on apprend des accords pour ça.

Apprendre des accords devient bien moins pénible, voire agréable si c’est
par et pour une chanson.

\subsection{Connaître les notes qui composent un accord}

Tout cela est lié à la position des touches noires sur un clavier, par exemple
de piano. Voir un cours d’harmonie pour plus de précisions.

\subsection{Majeur}

\begin{tabular}{ | c | c | c | c | c | c | c | c | }
    \hline
    Fond. & Do & Ré      & Mi      & Fa & Sol & La      & Si \\
    \hline
    3ce   & Mi & Fa\shrp & Sol\shrp & La & Si & Do\shrp & Ré\shrp \\
    \hline
    5te   & Sol & La     & Si      & Do & Ré & Mi      & Fa\shrp \\
    \hline
\end{tabular}

Ou avec les noms anglo-saxons~:

\begin{tabular}{ | c | c | c | c | c | c | c | c | }
    \hline
    Fond. & C & D      & E      & F & G & A      & B \\
    \hline
    3ce   & E & F\shrp & G\shrp & A & B & C\shrp & D\shrp \\
    \hline
    5te   & G & A      & B      & C & D & E      & F\shrp \\
    \hline
\end{tabular}

\subsection{Mineur}

\begin{tabular}{ | c | c | c | c | c | c | c | c | }
    \hline
    Fond. & Do        & Ré & Mi & Fa        & Sol        & La & Si \\
    \hline
    3ce   & Mi$\flat$ & Fa & Sol & La$\flat$ & Si$\flat$ & Do & Ré \\
    \hline
    5te   & Sol       & La & Si & Do        & Ré        & Mi & Fa\shrp \\
    \hline
\end{tabular}

Ou avec les noms anglo-saxons~:

\begin{tabular}{ | c | c | c | c | c | c | c | c | }
    \hline
    Fond. & C        & D & E & F        & G        & A & B \\
    \hline
    3ce   & E$\flat$ & F & G & A$\flat$ & B$\flat$ & C & D \\
    \hline
    5te   & G        & A & B & C        & D        & E & F\shrp \\
    \hline
\end{tabular}

\section{Accords à quatre sons}

\subsection{Le principe}

Les notes qui composent ces accords se retrouvent de la même façon que pour
trois sons. Il y a les accords «~M7~» (parfois notés $\Delta$), «~m7~», «~7~»
et «~m7b5~» (parfois notés $\emptyset$).

\subsection{Accords au sillet}

\ldots

\subsection{Accords avec barrés}

\ldots

\subsection{Drop 2 sur les quatre premières cordes}

C’est moins pratique sur les autres cordes, mais instructif.

\ldots

\section{Autres accords}

\gtab{E&69}{5:X2(11)22:020034}

\section{Contact} \label{contact}

\setlength{\parindent}{0pt}

Site~: \url{https://grahack.github.io/accords_guitare/} \\
ou plus simplement \url{https://huit.re/acc}

Contact~: \url{profgra.org@gmail.com}

\end{document}
